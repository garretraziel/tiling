\documentclass[a4paper, 11pt]{article}
\usepackage[czech]{babel}
\usepackage[utf8]{inputenc}
\author{Jan Sedlák}
\title{Vytváření dopředné neuronové sítě pomocí algoritmu Tiling}
\frenchspacing
\begin{document}
\maketitle
\section*{Úvod}
Následující text se zabývá použitím algoritmu \emph{Tiling} pro vytváření neuronových sítí s proměnnou topologií. U tradičních neuronových sítí, které mají topologii předem pevně danou, je důležitou vlastností algoritmus pro správné nastavování vah spojů, učení. Mezi nejznámější patří například algoritmus \emph{Backpropagation}, který pro zmenšování chyby sítě používá zpětné šíření chyby.

U neuronových sítí s pevně danou topologií můžeme narazit na mnoho problémů. Při učení sítě algoritmem Backpropagation musíme předem vědět, kolik musí mít síť skrytých vrstev a kolik má být neuronů v každé vrstvě. Jedná se však o netriviální úlohu a její řešení má výrazný vliv na kvalitu sítě. Řešením tohoto problému je použití algoritmu pro vytváření sítě s proměnnou topologií. Tyto algoritmy začínají budovat síť od jednoho jediného neuronu a postupně přidávají neurony v téže vrstvě, případě při\-dá\-va\-jí vrstvy nové, dokud není síť schopna daný problém vyřešit. Výsledkem takovýchto algoritmů jsou díky tomu jednoduché a dostatečně malé sítě.

Jedním z algoritmů pro vytváření sítě s proměnnou topologií je algoritmus Tiling~\cite{mezard}. Při vytváření sítě algoritmem Tiling vzniká dopředná neuronová síť složená z neuronů s bipolárním výstupem. První neuron v každé vrstvě je považován za tzv.~\emph{Master} neuron, který se snaží snížit počet chyb klasifikace předchozí vrstvy. Další neurony v každé vrstvě jsou tzv.~\emph{Ancillary} neurony, jejichž úkolem je rozlišit od sebe data, patřící do různých tříd, pokud je všechny dosavadní neurony v aktuální vrstvě přiřadily do stejné třídy. Algoritmus Tiling je určený pouze pro sítě klasifikujících do dvou tříd, ačkoliv existuje jeho úprava nazvaná \emph{MTiling}~\cite{mtiling} pro klasifikování do libovolného počtu tříd.

V rámci práce byl algoritmus Tiling implementován a byla změřena úspěšnost jím vytvářených sítí na dvou různých problémech.

\section*{Rozbor použitých metod}

\subsection*{Algoritmus Tiling}
\subsection*{Pocket algoritmus}
\subsubsection*{Problém rozdělování tříd}
\section*{Koncepce}
\section*{Experimentální ověření výsledných sítí}
\section*{Popis ovládání výsledného programu}
\section*{Shrnutí výsledků a závěr}

\begin{thebibliography}{9}
\bibitem{mezard}
  Mezard, M. and Nadal, Jean-P.,
  \emph{Learning in feedforward layered networks: The tiling algorithm}.
  Journal of Physics A: Mathematical and General,
  1989.
\bibitem{mtiling}
  Yang, J and  Parekh, R. G. and Honavar, V.,
  \emph{MTiling - A Constructive Neural Network Learning Algorithm for Multi-Category Pattern Classification}.
  Proceedings of the World Congress on Neural Networks'96,
  1996.
\end{thebibliography}
\end{document}